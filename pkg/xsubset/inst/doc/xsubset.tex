\documentclass[nojss]{jss}
\usepackage{thumbpdf}
%% need no \usepackage{Sweave}

%% additional commands
\newcommand{\squote}[1]{`{#1}'}
\newcommand{\dquote}[1]{``{#1}''}
\newcommand{\fct}[1]{{\texttt{#1()}\index{#1@\texttt{#1()}}}}
\newcommand{\class}[1]{\dquote{\texttt{#1}}}

\author{Marc Hofmann\\Universit{\'e} de Nech{\^a}tel
   \And Cristian Gatu\\Universit{\'e} de Nech{\^a}tel
   \AND Erricos J. Kontoghiorghes\\Birbeck College
   \And Achim Zeileis\\WU Wirtschaftsuniversit\"at Wien}
\Plainauthor{Marc Hofmann, Cristian Gatu, Erricos J. Kontoghiorghes, Achim Zeileis}

\title{\pkg{xsubset}: Efficient Computation of Best Subset Linear Regressions in \proglang{R}}
\Plaintitle{xsubset: Efficient Computation of Best Subset Linear Regressions in R}

\Keywords{linear regression, model selection, variable selection, best subset regression, \proglang{R}}
\Plainkeywords{linear regression, model selection, variable selection, best subset regression, R}


%\VignetteIndexEntry{xsubset: Efficient Computation of Best Subset Linear Regressions in R}
%\VignetteDepends{stats}
%\VignetteKeywords{linear regression, model selection, variable selection, best subset regression, R}
%\VignettePackage{xsubset}

\Abstract{ 
  todo
}

\Address{
  Marc Hofmann\\
  Institut d'Informatique\\
  Universit{\'e} de Nech{\^a}tel\\
  Rue Emile-Argand 11\\
  CP 158 -- 2009 Nech{\^a}tel, Switzerland\\
  E-mail: \email{Marc.Hofmann@unine.ch}\\
}

\begin{document}


\section{Introduction} \label{sec:intro}

introduce the \pkg{xsubset} package \citep{xsubset:Hofmann+Gatu+Kontoghiorghes:2009}
for the \proglang{R} system for statistical computing \citep{xsubset:R:2009},
available from the Comprehensive \proglang{R} Archive Network at
\url{http://CRAN.R-project.org/package=xsubset}

implements the algorithms of \cite{xsubset:Hofmann+Gatu+Kontoghiorghes:2007}
and \cite{xsubset:Gatu+Kontoghiorghes:2006}


somewhere we need to comment on alternative approaches in R, especially
the \pkg{leaps} package \citep{xsubset:Lumley+Miller:2009} based on \cite{xsubset:Miller:2002}
and the \pkg{subselect} package \citep{xsubset:OrestesCerdeira+DuarteSilva+Cadima:2009}
base on \cite{xsubset:DuarteSilva:2001}


\section{Implementation} \label{sec:implementation}

one could probably briefly comment on the algorithm and then
point out how it is implemented in \proglang{C++}

\subsection[C++]{\proglang{C++}}

\subsection[R interface]{\proglang{R} interface}

\proglang{S}3 class \class{xsubset} with associated methods

\fct{xsubset} is generic with the default method being the workhorse 
function and taking the regressor matrix \code{x} and response vector \code{y}
as main arguments. In addition, a \class{formula} and a \class{lm} method
are provided for convenience.

Many methods that work for \class{lm} objects also work for \class{xsubset}
objects. The idea is always that the linear model is refitted for the
specified \code{size} and then the information is extracted. If some method
is not available, \fct{refit} can always be called explicitly which returns
a \class{lm} object. For some
methods, \code{size} can also be a vector of sizes that should be assessed.
The default for \code{size} is typically the best BIC model, sometimes (as
in the \fct{summary} and \fct{plot} method) all sizes available in the 
\class{xsubset} object are assessed.  Currently, the methods to the generic functions include
\fct{print}, \fct{summary}, \fct{coef}, \fct{vcov}, \fct{logLik}, \fct{residuals}, 
\fct{fitted}, \fct{model.frame}, \fct{model.matrix}, \fct{AIC}, \fct{logLik}, \fct{deviance}.  


\section{Illustrations} \label{sec:illustrations}

load package and example data, for convenience already take logs for relative potentials
%
\begin{Schunk}
\begin{Sinput}
R> library("xsubset")
R> data("AirPollution", package = "xsubset")
R> for(i in 12:14) AirPollution[[i]] <- log(AirPollution[[i]])
\end{Sinput}
\end{Schunk}
%
then fitting best subsets can be done via
%
\begin{Schunk}
\begin{Sinput}
R> xs <- xsubset(mortality ~ ., data = AirPollution)
R> xs
\end{Sinput}
\begin{Soutput}
Call:
xsubset(object = mortality ~ ., data = AirPollution)
                           
Number of regressors:  16  
Include:               1   
Exclude:               -   
Subset sizes assessed: 2:16
Intercept:             YES 
\end{Soutput}
\end{Schunk}
%
To obtain a more complete picture, look at visualization (see
Figure~\ref{fig:summary}) or a tabular summary:
%
\begin{Schunk}
\begin{Sinput}
R> plot(xs)
R> summary(xs)
\end{Sinput}
\begin{Soutput}
Call:
xsubset(object = mortality ~ ., data = AirPollution)

Selected variables:
              2 3 4 5 6 7 8 9 10* 11 12 13 14 15 16
+(Intercept)  x x x x x x x x x   x  x  x  x  x  x 
precipitation     x x x x x x x   x  x  x  x  x  x 
temperature1        x x x x x x   x  x  x  x  x  x 
temperature7                  x   x  x  x  x  x  x 
age                         x x   x  x  x  x  x  x 
household                 x x x   x  x  x  x  x  x 
education       x     x x x x x   x  x  x  x  x  x 
housing                                 x  x  x  x 
population                           x  x  x  x  x 
noncauc       x x x x x x x x x   x  x  x  x  x  x 
whitecollar                                      x 
income                                     x  x  x 
hydrocarbon             x x x x   x  x  x  x  x  x 
nox                 x x x x x x   x  x  x  x  x  x 
so2               x               x  x  x  x  x  x 
humidity                                      x  x 

Model fit:
           2       3       4       5       6       7       8       9
AIC    638.8   623.3   610.2   600.6   597.1   596.9   596.6   596.3
RSS 133694.5 99841.1 77673.5 64037.8 58390.6 56314.6 54128.4 52101.6
        10*      11    12      13      14      15      16
AIC   594.1   594.7   596   597.3   599.2   601.1   603.1
RSS 48610.2 47471.4 46894 46380.2 46280.2 46248.6 46248.6
\end{Soutput}
\end{Schunk}


\begin{figure}{t}
\begin{center}
\includegraphics{xsubset-summary}
\caption{\label{fig:summary} BIC and RSS.}
\end{center}
\end{figure}

extract information (by default for best BIC model)
\begin{Schunk}
\begin{Sinput}
R> deviance(xs)
\end{Sinput}
\begin{Soutput}
        2         3         4         5         6         7         8 
133694.54  99841.07  77673.52  64037.82  58390.63  56314.60  54128.39 
        9        10        11        12        13        14        15 
 52101.56  48610.18  47471.39  46893.66  46380.24  46280.17  46248.62 
       16 
 46248.59 
\end{Soutput}
\begin{Sinput}
R> logLik(xs)